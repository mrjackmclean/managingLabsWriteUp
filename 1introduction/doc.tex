\chapter{Introduction}
The following section outlines the background of this MSc dissertation. TODO ??? The existing methods and tools shall justify the motivation behind this project. 

\section{Background}

The School of Computer Science at the University of St.\ Andrews has historically run lab sessions for early years students, offering them support in developing their practical programming skills. The school traditionally operated these labs using a ‘hands-up’ system to organise in-class support, however the school's relatively rapid growth in class size means that this method is becoming less efficient. The global pandemic has further accelerated the need for an alternative support system, since the in-person labs replacement with online sessions meant that the traditional system was not possible. 


\section{Objectives}

The overall goal of this project is to create a custom system for the School that can be used to manage students' requests for practical programming assistance, both in virtual and physical lab sessions. Objectives for the project are listed below.

\subsection{Primary}
\begin{itemize}
    \item Investigate and evaluate existing methods of managing labs, identifying weak points.
    \item Undertake established software engineering techniques to plan and design the lab management system.
    \item Create a system on which:
    \begin{itemize}
        \item Students are able to create `tickets' to request help from class demonstrators through a web interface.
        \item Class demonstrators are able to assign, comment on, resolve and close `tickets'.
        \item Staff are able to view statistics and summaries of activity - for example which `tickets' were resolved by which demonstrator, the time of resolution and the amount of time the `tickets' were open.
    \end{itemize}
    \item Evaluate a prototype of the system using volunteer sets of class demonstrators, then complete a final evaluation without external parties. 
    
\end{itemize}

\subsection{Secondary}
\begin{itemize}
    \item Implement graphical representation of statistics and summaries generated by the system.
    \item Optimise the system for use on mobile devices. 
    \item Create a UI recommending anonymised `tickets' to students that may be similar to the their request.
    \item Implement a live text chat element on each ticket, allowing interaction between the demonstrator and student.
\end{itemize}

\subsection{Tertiary}
\begin{itemize}
    \item Implement authentication on the system using the Shibboleth single sign-on system used by the University of St. Andrews.
    \item Implement a method to allow students and class demonstrators to communicate by video chat. 
\end{itemize}