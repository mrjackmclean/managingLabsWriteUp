\chapter{Introduction}

The School of Computer Science at the University of St.\ Andrews has historically run lab sessions for early years students, offering them support in developing their practical programming skills. The school traditionally operated these labs using a ‘hands-up’ system to organise in-class support, however the school's relatively rapid growth in class size means that this method is becoming less efficient. The global pandemic has further accelerated the need for an alternative support system, since the in-person labs replacement with online sessions meant that the traditional system was not possible. 

TODO - add more, allude to stuff in context survey and  give outline of dissertation structure

\section{Objectives}

The overall goal of this project is to create a custom system for the School that can be used to manage students' requests for practical programming assistance, both in virtual and physical lab sessions. Objectives for the project are listed below. 

TODO discuss which were achieved

The main contribution of the dissertation to the field was providing a tool that allows staff to effectively manage the provision of one-to-one assistance for both in-person and remote settings. TODO expand

\subsection{Primary}
\begin{itemize}
    \item [PO1] Investigate and evaluate existing methods of managing labs, identifying weak points.
    \item [PO2] Undertake established software engineering techniques to plan and design the lab management system.
    \item [] Create a system on which:
    \begin{itemize}
        \item [PO3.1] Students are able to create `tickets' to request help from class demonstrators through a web interface.
        \item [PO3.2] Class demonstrators are able to assign, comment on, resolve and close `tickets'.
        \item [PO3.3] Staff are able to view statistics and summaries of activity - for example which `tickets' were resolved by which demonstrator, the time of resolution and the amount of time the `tickets' were open.
    \end{itemize}
    \item [PO4] Evaluate a prototype of the system using volunteer sets of class demonstrators, then complete a final evaluation without external parties. 
    
\end{itemize}

\subsection{Secondary}
\begin{itemize}
    \item [SO1] Implement graphical representation of statistics and summaries generated by the system.
    \item [SO2] Optimise the system for use on mobile devices. 
    \item [SO3] Create a UI recommending anonymised `tickets' to students that may be similar to the their request.
    \item [SO4] Implement a live text chat element on each ticket, allowing interaction between the demonstrator and student.
\end{itemize}

\subsection{Tertiary}
\begin{itemize}
    \item [TO1] Implement authentication on the system using the Shibboleth single sign-on system used by the University of St. Andrews.
    \item [TO2] Implement a method to allow students and class demonstrators to communicate by video chat. 
\end{itemize}