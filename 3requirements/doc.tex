\chapter{Requirements Specification}
\label{chap:req}
In this chapter, system requirements are outlined. The first section shows a \gls{uml} use case diagram, used to explore and convey uses of the system in order to help derive requirements. The requirements were derived through discussion with the customer, analysis of the strengths and weaknesses of the current system and findings from research and investigation into similar systems, such as Spiceworks and ClassroomQ which are discussed in chapter \ref{chap:context}. The second section defines the requirements outright.

\section{Use Case}
\subsection{Use Case Diagram}

The following section contains a UML use case diagram, show in fig \ref{fig:useCase}, based on the requirements for the proposed `codeHelper' application. The diagram models `full' interactions, interactions with the system as well as some output from the system \cite{uml}.

\begin{figure}[H]
    \centering
    \includegraphics[width=0.9\textwidth]{3requirements/images/useCase.png}
    \caption{Use case diagram for lab management system.}
    \label{fig:useCase}
\end{figure}

\begin{comment}

\subsection{Use Case Specifications}

The following section outlines some use case specifications/descriptions for some of the use cases in the use case diagram show in figure \ref{fig:useCase}. This helps to outline detail so as to help understand exactly how the system's concerns are met \cite{uml}. 

\subsection*{`Create Ticket' Use Case Specification}
\begin{table}[H]
\centering
 \begin{tabular}{p{0.27\linewidth}  p{0.67\linewidth}}
 \textbf{Use case name} & \textbf{Post Ticket}  \\
 Use case ID & 1\\
 Brief Description & A student creates a ticket on the system.\\
 Preconditions & The student is a registered, the lab is open.\\
 Primary actors & Student. \\
 Secondary actors & \textbf{None.} \\
 Successful end condition & A new live ticket is posted onto the system for class demonstrators to view and the student is redirected to the queue page. \\
 Failed end condition & The ticket listing is not posted. \\
 Main flow & Steps:\\
 & 1. The use case starts when a student tries to visit the `post a ticket' page.\\
 & 2. The system checks that the user is logged in as a registered account with student role privileges. \\
 & 3. The `post a ticket' page is opened. \\
 & 4. The student inputs the details of the issue - associated module, relevant practical or worksheet and an issue description.\\
 & 5. The student submits the information. \\
 & 6. The student selects existing tags or creates new tags for the ticket. \\
 & 7. The new ticket is posted on the application and the student is redirected to the queue page.\\

 Extension & Step\hspace{0.3cm} Branching Action \\
 & 2.1.1 \hspace{0.5cm}The user is not logged in. \\
 & 2.1.2 \hspace{0.5cm}The user is redirected to the login page. \\
 & 2.2.1 \hspace{0.5cm}The user is logged in as a demonstrator or lab lead. \\
 & 2.2.2 \hspace{0.5cm}The user is redirected to the helpdesk page. \\
 
\end{tabular}
\end{table}

\newpage
\subsection*{`Resolve Ticket' Use Case Specification}
\begin{table}[H]
\centering
 \begin{tabular}{p{0.27\linewidth}  p{0.67\linewidth}}
 \textbf{Use case name} & \textbf{Resolve Ticket}  \\
 Use case ID & 2\\
 Brief Description & A class demonstrator resolves a student's issue ticket.\\
 Preconditions & The ticket is live and not assigned to any class demonstrators.\\
 Primary actors & Class Demonstrator. \\
 Secondary actors & Student. \\
 Successful end condition & The ticket is marked as closed. \\
 Failed end condition & The ticket is not marked as closed. \\
 Main flow & Steps:\\
 & 1. The use case starts when a class demonstrator attempts to visit the live tickets page.\\
 & 2. The system checks that the user is logged in as a registered, verified account with class demonstrator role privileges. \\
 & 3. The live ticket page is opened and shows a queue of all unresolved, not closed tickets that were posted in the current lab. \\
 & 4. The class demonstrator selects a ticket in the queue.\\
 & 5. The class demonstrator assigns themselves to the ticket. \\
 & 6. The class demonstrator sends the student a live chat, either explaining the solution, asking for more information or requesting a video chat. \\
 & 7. The class demonstrator solves the student's problem and marks the issue ticket as resolved. \\
 & 8. The ticket is archived on the system.\\

 Extension & Step\hspace{0.3cm} Branching Action \\
 & 2.1.1 \hspace{0.5cm}The user is not logged in. \\
 & 2.1.2 \hspace{0.5cm}The user is redirected to the login page. \\
 & 2.2.1 \hspace{0.5cm}The user is logged in as a student. \\
 & 2.2.2 \hspace{0.5cm}The user is redirected to the position of the student page that they are at (post ticket, queue or chat). \\
 & 5.1 \hspace{0.5cm} The class demonstrator does not assign themselves but messages the student.\\
 & 5.2 \hspace{0.5cm} The class demonstrator is automatically assigned.\\
 & 5.3 \hspace{0.5cm} Steps 7-8.\\
 & 7.1 \hspace{0.5cm} The class demonstrator creates a reusable solution and attaches it to the ticket.\\
 & 7.2 \hspace{0.5cm} Steps 7-8.\\
\end{tabular}
\end{table}
\end{comment}

\section{System Requirements}

\paragraph{Stakeholders} The system has four different types of stakeholder - the students, the class demonstrator, the lab lead and the system administrator. The system administrator role was designed with the intention of being managed by course organisers.

\subsection{Functional Requirements}

The following requirements list is prioritised using the \gls{moscow} technique, reducing the indecision associated with simpler prioritisation. They are split with the view that no more than 60\% of effort should be spent on `Must Have' requirements of the project, along with a sensible pool of `Could Haves' of around 20\% effort \cite{dsdm}.

\begin{table}[H]
\small
\begin{tabular}{|p{0.05\linewidth} | p{0.78\linewidth} |p{0.09\linewidth}|}
 \hline
 \textbf{ID} & \textbf{Details} & \textbf{Priority} \\
 \hline
 
 \multicolumn{3}{c}{\textit{\textbf{Account Requirements}}}\\
 
 \hline
 F-01 & \textit{Description:} The system must allow students to register on the system by providing a username and a password. & M\\
  \cline{2-2}
  & \textit{Rationale:} Students must be able to sign up to the system. & \\

  
   \hline\hline
 F-02 & \textit{Description:} The system would allow students to sign in using the Shibboleth \gls{sso} system used by the school. & W\\
  \cline{2-2}
  & \textit{Rationale:} Students can only log in if Shibboleth authenticates them, making the system more secure, reliable and meaning authentication is centralised to align with other school systems. & \\

  
     \hline\hline
 F-03 & \textit{Description:} Lab leads must be able to assign class demonstrator or lab lead roles to users. & M\\
  \cline{2-2}
  & \textit{Rationale:} This allows class demonstrator accounts to be created without allowing regular students to attempt to class create demonstrator accounts. & \\
  
   \hline\hline
 F-04 & \textit{Description:} Users should allow students to log in and out of the system. & M\\
  \cline{2-2}
  & \textit{Rationale:} Students can only persist login and sessions on the system, as well as having past tickets, if they can log into the system. & \\
\hline
  
   \multicolumn{3}{c}{\textit{\textbf{Ticket Requirements}}}\\
  
 \hline
 F-05 & \textit{Description:} Whilst the lab is open, students must be able to create `tickets' by providing information on the module code, practical or workshop number, issue category and issue description. & M\\
  \cline{2-2}
  & \textit{Rationale:} This creates a record of the issue that the student is having for the class demonstrators to interact with. & \\

  
 \hline\hline
 F-06 & \textit{Description:} Students should be able to add tags to their own tickets which associate them with related tickets. & S\\
  \cline{2-2}
  & \textit{Rationale:} This allows students to give more clarification about the nature of the problem, create links with similar tickets (which they can view) and also help the system recommendation algorithm recommend similar past tickets. & \\

  
    \hline\hline
 F-07 & \textit{Description:} The system would recommend similar archived tickets before students post their tickets. & W\\
  \cline{2-2}
  & \textit{Rationale:} This allows students to scan archived tickets, checking if any will help resolve their issue, before they post a live ticket that demonstrators need to deal with. & \\
  
      \hline\hline
 F-08 & \textit{Description:} The system would keep a record of which archived tickets students have selected (`this solved my problem') before deleting the original ticket draft. & W\\
  \cline{2-2}
  & \textit{Rationale:} This allows lad leads and demonstrators to get information on issues which multiple students have had. & \\
  
   \hline\hline
 F-09 & \textit{Description:} Class demonstrators must be able to assign themselves to tickets. & M\\
  \cline{2-2}
  & \textit{Rationale:} This allows demonstrators to keep track of, and indicate to other demonstrators, which issues they are working on or have completed. & \\

  
  \hline\hline
 F-10 & \textit{Description:} Class demonstrators should be able to livechat on their own assigned tickets. & S\\
  \cline{2-2}
  & \textit{Rationale:} This allows demonstrators to discuss and initiate contact with the students. & \\

    \hline\hline
 F-11 & \textit{Description:} Class demonstrators should be able to close tickets. & S\\
  \cline{2-2}
  & \textit{Rationale:} This allows demonstrators to mark tickets as no longer being `live' - either marking them as resolved or closed. & \\

  
      \hline\hline
 F-12 & \textit{Description:} Students should be able to close tickets. & S\\
  \cline{2-2}
  & \textit{Rationale:} This allows students to remove their issues from the queue if they have solved the problem themselves. & \\
\hline



  \end{tabular}
\end{table}

\begin{table}[H]
\small
\begin{tabular}{|p{0.05\linewidth} | p{0.78\linewidth} |p{0.09\linewidth}|}
  
          \hline
 F-13 & \textit{Description:} Lab leads should be able to `open' labs, allowing tickets to be created by students (in that lab). & S\\
  \cline{2-2}
  & \textit{Rationale:} This allows lab leads to prevent help requests being posted outwith their lab hours. & \\

  
\hline\hline
 F-14 & \textit{Description:} Lab leads would be able to disable user accounts from the system. & W\\
  \cline{2-2}
  & \textit{Rationale:} This allows lab leads to remove users who are no longer students, demonstrators or alternatively for inappropriate behaviour or posting excessively. & \\
 
 \hline\hline
 F-15 & \textit{Description:} The system would not allow offensive or abusive content to be posted. & W\\
  \cline{2-2}
  & \textit{Rationale:} Prevents offensive behaviour. & \\
 
 \hline\hline
 F-16 & \textit{Description:} The system could not allow students to have more than one live ticket. & C\\
  \cline{2-2}
  & \textit{Rationale:} This prevents individual students from posting an excessive number of tickets on the system and also encourages students to prioritise the issues they ask for help with. & \\
 
  \hline\hline
 F-17 & \textit{Description:} The system would allow students to edit and update their live tickets. & W\\
  \cline{2-2}
  & \textit{Rationale:}  This allows users to provide more information on a ticket to improve demonstrators ability to help them. & \\
 
   \hline\hline
 F-18 & \textit{Description:} The system could have an integrated live chat feature on every live ticket. & C\\
  \cline{2-2}
  & \textit{Rationale:} This allows demonstrators to send messages on live chats for tickets that they are not assigned, helping out other demonstrators. & \\

   \hline\hline
 F-19 & \textit{Description:} The system could have some form of video chat option on each ticket that has a demonstrator assigned. & C\\
  \cline{2-2}
  & \textit{Rationale:} This allows students and demonstrators to resolve the problem raised by the ticket, removing any human error associated with looking the student up on another system - as well as reducing time spent doing so. & \\

   \hline\hline
 F-20a & \textit{Description:} The system could allow class demonstrators to create a `solution' for ticket. & C\\
  \cline{2-2}
  & \textit{Rationale:} This allows demonstrators to store solutions with archived tickets, useful for future reference when archived tickets are recommended as similar to future students' tickets. & \\
  
  \hline\hline
 F-20b & \textit{Description:} The system could allow class demonstrators to edit existing `solutions' for tickets. & C\\
  \cline{2-2}
  & \textit{Rationale:} This allows demonstrators to update and correct old solutions. & \\
  
    \hline\hline
 F-20c & \textit{Description:} The system could allow class demonstrators to assign existing `solutions' to tickets. & C\\
  \cline{2-2}
  & \textit{Rationale:} This allows demonstrators to find and assign preexisting solutions to new tickets, reducing the time spent answering frequently asked questions. & \\

   \hline\hline
 F-21 & \textit{Description:} The system would prompt class demonstrators to provide solutions for archived tickets which have repeatedly been marked by students as similar to their own tickets. & W\\
  \cline{2-2}
  & \textit{Rationale:} This encourages demonstrators to provide solutions for more common problems, ensuring the recommendation system will be useful and therefore reduce the number of new tickets being created. It is also useful for reference by other class demonstrators. & \\

     \hline
 
 \end{tabular}
\end{table}
 
 \begin{table}[H]
\small
\begin{tabular}{|p{0.05\linewidth} | p{0.78\linewidth} |p{0.09\linewidth}|}
 
 
 \hline
 F-22 & \textit{Description:} The system should give some indication to students of how busy the current lab session is or how many tickets are ahead of them in the queue. & S\\
  \cline{2-2}
  & \textit{Rationale:} This allows students to appreciate the amount of wait time that a response may require. & \\

     \hline\hline
 F-23 & \textit{Description:} The system could allow students to specify their location for in-person labs. & C\\
  \cline{2-2}
  & \textit{Rationale:} This allows the tool to be used for management of in-person labs, as well as labs which are both in-person and have virtual participants. & \\
 
      \hline\hline
 F-24 & \textit{Description:} The system could allow students to send files to demonstrators. & C\\
  \cline{2-2}
  & \textit{Rationale:} This allows demonstrators to study, understand and potentially solve issues more quickly than they would just by chatting to the student. & \\
  \hline
 
  \multicolumn{3}{c}{\textit{\textbf{Summary Requirements}}}\\
  
          \hline
 F-25 & \textit{Description:} The system could show lab leads a `timeline' of all actions on the system - such as lab opening, ticket posting, ticket resolution and lab closing. & C\\
  \cline{2-2}
  & \textit{Rationale:} This allows demonstrators and lab leads to quickly scan recent activity, as well as helping demonstrators to quickly find recent tickets or events. & \\
  
\hline\hline
 F-26 & \textit{Description:} Lab leads must be able to view lab summaries which provide statistics about the amount of tickets resolved, who resolved tickets and how long tickets remained unresolved in each individual lab. & M\\
  \cline{2-2}
  & \textit{Rationale:} This allows lab leads to review the efficiency of the ticketing system in the lab. & \\

 \hline\hline
 F-27 & \textit{Description:} Lab leads must be able to view class summaries which provide statistics about the amount of tickets resolved, who resolved tickets and how long tickets remained unresolved for all labs in a given class. & M\\
  \cline{2-2}
  & \textit{Rationale:} This allows lab leads to review the efficiency of the ticketing system in the class. & \\
  \hline
  
\end{tabular}
\end{table}

\subsection{Non-Functional Requirements}

\begin{table}[H]
\small
\begin{tabular}{|p{0.07\linewidth} | p{0.78\linewidth} |p{0.09\linewidth}|}
 \hline
 \textbf{ID} & \textbf{Details} & \textbf{Priority} \\
 
 \hline
   \multicolumn{3}{c}{\textit{\textbf{Usability Requirements}}}\\
 \hline
 
   NF-01 & \textit{Description:} 95\% of students should be able to create a ticket in less than 5 minutes on the first attempt. & S \\
  \cline{2-2}
  & \textit{Rationale:} The system should be intuitive, quick and easy to use. & \\

   \hline\hline
      NF-02 & \textit{Description:} 95\% of the time, a class demonstrator should be able to mark a ticket as resolved within 1 minute of providing the resolution to the student & S \\
  \cline{2-2}
  & \textit{Rationale:} The system should be intuitive, quick and easy to use. & \\
   \hline
     
     \multicolumn{3}{c}{\textit{\textbf{Security Requirements}}}\\
     
     \hline
 NF-03 & \textit{Description:} Passwords must be encrypted. & M\\
  \cline{2-2}
  & \textit{Rationale:} Data needs to be protected from hostile users. & \\
  
    \hline\hline
 NF-04 & \textit{Description:} Data must be stored securely. & M\\
  \cline{2-2}
  & \textit{Rationale:} Data needs to be protected from hostile users. & \\
  
      \hline\hline
 NF-05 & \textit{Description:} The system would anonymise tickets that are used in future recommendations. & W\\
  \cline{2-2}
  & \textit{Rationale:} This allows the system to show users the archived tickets without processing or storing any personal data. & \\
  
      \hline\hline
 NF-06 & \textit{Description:} The system must only allow class demonstrators (and lab leads) to view live tickets and names of students. & M\\
  \cline{2-2}
  & \textit{Rationale:} This addresses privacy concerns arising from students being able to view other students' names and tickets. & \\
\hline
  
    \multicolumn{3}{c}{\textit{\textbf{Performance Requirements}}}\\
    
  \hline
   NF-07 & \textit{Description:} Class demonstrator ticket assignment should be processed and updated for other users within 0.1s on a local host (to ignore network delays). & S \\
  \cline{2-2}
  & \textit{Rationale:} Slow assignment processing and updating could result in issues with conflicting assignment and demonstrators working on the same ticket. & \\
 
   \hline\hline
   NF-08 & \textit{Description:} Resolution of tickets should be processed and updated for other users within 1s on a local host (to ignore network delays). & S \\
  \cline{2-2}
  & \textit{Rationale:} Ticket resolution should be processed reasonably quickly to prevent lab leads following up tickets resolved by demonstrators or demonstrators following up tickets resolved by students. & \\
  \hline
  
      \multicolumn{3}{c}{\textit{\textbf{Dependability Requirements}}}\\
  
   \hline
   NF-09 & \textit{Description:} The System should achieve 99\% up time. & S \\
  \cline{2-2}
  & \textit{Rationale:} Since students rely on the labs for help and they are only open for a small amount of time, it is critical that down time during the lab window is reduced. & \\
\hline
   
\multicolumn{3}{c}{\textit{\textbf{Space Requirements}}}\\
   
   \hline
   NF-10 & \textit{Description:} The system should be able to store at least 100,000 tickets. & S \\
    \cline{2-2}
  & \textit{Rationale:} Since there are many labs for each module, which involves multiple practicals, it is important that the archive is capable of storing tickets for future reference. & \\
 \hline

\end{tabular}
\end{table}


