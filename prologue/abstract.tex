\begin{abstract}

The School of Computer Science has traditionally run in-person lab sessions for sub-honours students to support them developing their practical programming skills. With the national lockdowns caused by the global pandemic, these in-person session had to be replaced with online sessions, which requires new processes to effectively manage and run the sessions. 

This dissertation reviews the current system as well as existing tools that could be adapted to manage virtual programming support sessions. It also provides the School with a custom tool for managing students' requests for practical programming assistance in virtual support sessions. These existing systems have significant weaknesses, so the custom tool \textit{codeHelper} was designed and developed to manage students' requests for practical programming assistance in virtual support sessions. A small scale evaluation of \textit{codeHelper} showed positive results, suggesting that such a tool would have a beneficial effect on virtual lab management. 

\end{abstract}
